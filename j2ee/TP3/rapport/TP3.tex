\documentclass[a4paper,12pt]{article} 
\usepackage[T1]{fontenc}              
\usepackage[frenchb]{babel} % césures, titres français
\usepackage[utf8]{inputenc} % encodage
\usepackage[a4paper,left=3cm,right=3cm,top=2cm,bottom=2cm]{geometry} % marges
\usepackage{graphicx} % insertion d'images
\usepackage{rotating}
\usepackage{float} % permet d'utiliser H pour placer un flottant obligatoirement
\usepackage{pdfpages} % inclusion de PDF au sein du document
\usepackage{listings}
\pagestyle{plain} % pied de pages simples

\setlength{\parskip}{1ex plus 0.5ex minus 0.2ex} % espace entre les paragraphes
\setcounter{tocdepth}{2}
\setcounter{secnumdepth}{2}

\lstset{% general command to set parameter(s)
basicstyle=\ttfamily, % print whole listing small
keywordstyle=\color{black}\bfseries\underbar,
% underlined bold black keywords
identifierstyle=, % nothing happens
commentstyle=\color{white}, % white comments
showstringspaces=false,
numbers=left,
language=java,
breaklines=true,
frame=tblr} % no special string spaces

%%%% debut macro %%%%
\makeatletter
\renewcommand\section{\@startsection {section}{1}{\z@}%
                           {-3.5ex \@plus -1ex \@minus -.2ex}%
                           {2.3ex \@plus.2ex}%
                           {\normalfont\Large\bfseries}}
\makeatother
%%%% fin macro %%%%



% Def
\newcommand{\code}[1]{{\lstinline{#1}}}

\begin{document}
\newpage
\title{J2EE\\TP 3}
\date{}
\author{BRIZAI Olivier\\THORAVAL Maxime}
\maketitle


\section{Utilisation de Spring}
Dans la première partie du TP, nous avons appris à utiliser les bases de Spring. Ce dernier va nous permettre de dissocier, du code Java, 
les classes de service et les DAO. Ceci aura pour effet une plus grande facilité à changer de méthode récupération des données.\\
Par exemple: Nous pourrons décider de passer de l'utilisation d'une base de données à celle des fichiers juste en modifiant un fichier XML 
(en supposant que les classes nécessaires ont été crées au préalable).

\subsection{DAO}
Avant d'utiliser Spring, la récupération d'une classe DAO se faisait de cette manière 
\begin{lstlisting}
	DAOImpl dao = new DAOImpl();
	dao.init();
\end{lstlisting}
Ici, nous avons instancié un objet de type \textit{DAOImpl} et l'avons initialisé.\\
Le principal problème est lié au fait que nous utilisons une classe définie 


\end{document}



