\documentclass[a4paper,12pt]{article} 
\usepackage[T1]{fontenc}              
\usepackage[frenchb]{babel} % césures, titres français
\usepackage[utf8]{inputenc} % encodage
\usepackage[a4paper,left=3cm,right=3cm,top=2cm,bottom=2cm]{geometry} % marges
\usepackage{graphicx} % insertion d'images
\usepackage{rotating}
\usepackage{float} % permet d'utiliser H pour placer un flottant obligatoirement
\usepackage{pdfpages} % inclusion de PDF au sein du document
\usepackage{listings}
\pagestyle{plain} % pied de pages simples

\setlength{\parskip}{1ex plus 0.5ex minus 0.2ex} % espace entre les paragraphes
\setcounter{tocdepth}{2}
\setcounter{secnumdepth}{2}

\lstset{% general command to set parameter(s)
basicstyle=\ttfamily, % print whole listing small
keywordstyle=\color{black}\bfseries\underbar,
% underlined bold black keywords
identifierstyle=, % nothing happens
commentstyle=\color{white}, % white comments
showstringspaces=false,
numbers=left,
language=java,
breaklines=true,
frame=tblr} % no special string spaces

%%%% debut macro %%%%
\makeatletter
\renewcommand\section{\@startsection {section}{1}{\z@}%
                           {-3.5ex \@plus -1ex \@minus -.2ex}%
                           {2.3ex \@plus.2ex}%
                           {\normalfont\Large\bfseries}}
\makeatother
%%%% fin macro %%%%



% Def
\newcommand{\code}[1]{{\lstinline{#1}}}

\begin{document}
\newpage
\title{TP Recherche d'Information}
\date{}
\author{BRIZAI Olivier\\THORAVAL Maxime}
\maketitle

\newpage
\section{Introduction}
Dans ce TP nous nous sommes intéressé à la mise en place d'une méthode d'indexation. La méthode d'indexation utilisée ici est fondée sur la facteur TF-IDF et le coefficient de Salton.

Le coefficicient TF-IDF permet de pondérer un mot dans le document auquel il appartient. Ce coefficient ne dépend pas seulement du nombre d'apparition du mot dans le document en question mais également de
son apparition dans les autres documents. Une des propriétées remarquable de ce coefficient est donc justement de devenir très faible pour un mot donné, dans un index donné, si celui-ci est équi-réparti dans les documents de l'index. L'intérêt est alors de faire "disparaître" les mots très fréquemmentd utilisées dans les phrases comme des pronoms personnels, conjonction de coordination, pronoms relatifs ...

Le coefficient de Salton permet quant à lui de classer les documents d'un index par pertinence lors d'une requête. Il se base pour cela sur une méthode dite vectorielle. L'objectif étant de calculer la "distance" d'une requete à chaque document de l'index et de classer ensuite les documents suivant cette distance. Le document le plus "proche" de la requete étant alors le plus pertinant. Plusieurs méthode du calcul de cette distance sont envisageables.

\section{Choix de la structure}



\section{La phase d'indexation}

\section{La phase de recherche}

\section{La phase de recherche}

\section{Bilan du TP}
\end{document}


